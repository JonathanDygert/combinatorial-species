\documentclass{article}

\usepackage{main}
\usepackage{species-paper}

\title{Combinatorial Species}
\titleheader{Research Paper Based on \\ \citetitle{yorgey}}
\author{Jonathan Dygert}

\begin{document}
\maketitle

\section{Introduction}
Combinatorial species were introduced by French mathematician Andr\'e Joyal in his 1981 paper,
\citetitle{joyal}, to unify branches of combinatorics. The theory of combinatorial species is still
young, and the community is not yet in consensus on its utility.

Conveniently, it provides a great mathematical representation for data structures. This is of great
use to computer scientists, especially when working with expressive type systems. This notation
allows us to easily compare different structures, reason about their possible contents, perform
operations such as differentiation, and develop transformations between structures.

Consider a linked list, which can be represented as \( L = 1 + X \bullet L \). This can be read as
follows:
\[
  \underbrace{L}_\text{A list} \quad
  \underbrace{=}_\text{is} \quad
  \underbrace{1}_\text{empty} \quad
  \underbrace{+}_\text{or} \quad
  \underbrace{X}_\text{an element} \quad
  \underbrace{\bullet}_\text{and} \quad
  \underbrace{L}_\text{another list}
\]

We can expand this recursive definition by replacement and algebraic manipulation to get an
equivalent expression \( L = 1 + X + X^2 + X^3 + \cdots \). This tells us that the list can have 0
elements, 1 element, 2 elements, and so on. This also shows us that a linked list is equivalent
(more specifically, isomorphic) to an array-based list, which stores elements consecutively.

In addition to its possible utilities for combinatorics, this framework is invaluable for
investigating the fundamental properties of data structures, and serves to better strengthen the
understanding of types in computer science. Additionally, it has applications in automated testing,
by providing a method to generate inputs to test on, either randomly or by exhaustion.~\cite{yorgey}

\section{Combinatorial Species}
\subsection{What is a species?}
Informally, a species is \textcquote{yorgey}{a family of structures parameterized by a set of labels
which identify locations in the structures.} A species maps a given finite set of labels into a set
of structures containing those labels. The species must be ignorant of the contents of the labels,
and it must be possible to relabel without changing the structure itself.

\begin{definition}
  A \emph{species} \( F \) is a pair of mappings, \( F_\bullet \) and \( F_\leftrightarrow \), where
  \begin{itemize}
    \item \( F_\bullet \) maps a finite set \( U \) to a finite set of structures, \( F_\bullet [U]
      \), which can be ``built from'' the labels, commonly called the set of \( F \)-structures over
      \( U \).
    \item \( F_\leftrightarrow \) ``lifts'' a bijection \( \sigma : U_1 \leftrightarrow U_2 \)
      between two label sets onto a bijection between \( F \)-structures,
      \[
        F_\leftrightarrow [\sigma] : F_\bullet [U_1] \leftrightarrow F_\bullet [U_2]
      \]
      This mapping must be functorial. That is, it should satisfy
      \begin{itemize}
        \item \( F_\leftrightarrow [id] = id \)
        \item \( F_\leftrightarrow [\sigma_1 \circ \sigma_2]
          = F_\leftrightarrow [\sigma_1] \circ F_\leftrightarrow [\sigma_2] \)
      \end{itemize}
      where \( id \) is the identity mapping and \( \circ \) is composition.
  \end{itemize}~\cite{bergeron, yorgey}
\end{definition}

\( F_\bullet \) is typically written as \( F \) for simplicity. Because \( F_\leftrightarrow \) is
functorial, the values of the labels are unimportant. The important property of the labels is that
they are distinct, so that we can distinguish different locations of the structure.

A species corresponds to a data structure generic over a single unrestricted type parameter, such as
Java's \verb|LinkedList<T>|. However, the labels do not correspond to the data held by the data
structure. Instead, they should be thought of as names for the locations within the structure.

\subsection{Basic Species}
\subsubsection*{Zero}
The species \( 0 \) yields no structures no matter what labels it is given. This corresponds to a
type which cannot be created. This is less commonly encountered in programming, but formally is the
type of infinite loops or a function that terminates the program, and it can also be useful in
generic contexts to eliminate the variant of a sum type.

\subsubsection*{One}
The species \( 1 \) yields a single structure when given no labels. In other words, it holds no
data, but nonetheless can be created. This corresponds to the return type \verb|void| in Java, which
can be thought of as nothing, or more precisely, a unit type that can always be created from
nothing.

\subsubsection*{Singleton}
The species \( X \) yields a single structure when given a single label. In other words, it holds
exactly one piece of data. Conceptually, this corresponds to a type that is no different from the
type it is holding.

\subsection{Operators}
\subsubsection*{Sum}
Informally, species sum represents the word ``or''. We present a choice between two species.
Species sum corresponds to a type sum or disjoint (tagged) union. Given species \( F \) and \( G \),
and the set of labels \( U \), the structures of \( F + G \) over \( U \) are each either an \( F
\)-structure or a \( G \)-structure along with a tag that specifies which. Formally,
\[
  (F + G)[U] = F[U] \uplus G[U]
\]
where \( \uplus \) is disjoint union over sets.~\cite{yorgey}

We can also generalize \( 0 \) and \( 1 \) to any positive natural number \( n \) by making \( n \)
a species that takes no labels and produces \( n \) distinct structures. A simple way to define \( n
\) is as the species sum of \( n \) total \( 1 \) species. As an example, a Boolean (with values
true and false) would correspond to the species \( 1 + 1 = 2 \).

\subsubsection*{Product}
Informally, species product represents the word ``and''. We will take a structure from both species.
The product of species corresponds to type product, or pairing. Given species \( F \) and \( G \),
and set of labels \( U \), \( (F \bullet G)[U] \) is the set of all pairs of structures with \( U \)
split between \( F \) and \( G \). That is,
\[
  (F \bullet G)[U] = \Set{ (x, y) | U_1 \uplus U_2 = U,\, x \in F[U_1],\, y \in G[U_2] }
\]
A structure of \( (F \bullet G)[U] \) will be a pair of structures, one from \( F \) and one from \(
G \), with the labels of \( U \) split between the pair.

\subsection{Properties}
Species addition and multiplication has the typical properties we expect of these operations.  They
are associative, commutative, and have identities of zero and one, respectively.  Multiplication
distributes over addition, and zero is an annihilator for multiplication. These properties are not
proven here in interest of space.

\subsection{Regular Species}
A regular species can be defined equivalently by either its construction or its symmetry.

The species we have seen so far are all regular. By using using addition and multiplication, we can
construct more regular species. We are also implicitly using the least fixed point operator \(\mu \)
when we refer to a species in its definition. For example, \( L = 1 + X \bullet L \) would be
written as \( L = \mu \ell. 1 + X \bullet \ell \) when using the least fixed point operator. These
are equivalent due to the implicit species theorem, which is omitted here for space.~\cite{yorgey}

\begin{definition}
  A species is \emph{regular} if it can be expressed in terms of \( 0 \), \( 1 \), \( X \), \( + \),
  \( \bullet \), and \( \mu \) (self-reference).~\cite{yorgey}
\end{definition}

A species is also regular if it has no non-trivial symmetries. In other words, all of its possible
structures will change if the labels are rearranged.

\begin{definition}
  A species \( F \) is \emph{regular} if, for all structures \( f \in F[U] \), the only permutation
  \(\sigma \) such that \( F_\leftrightarrow [\sigma] (f) = f \) is the identity
  permutation.~\cite{yorgey}
\end{definition}

This definition involving symmetry is often more useful for testing if a species defined by some
idea is regular.

\section{Irregular Species}
\subsection{Simple Irregular Species}
\subsubsection*{Set}
\( E \) is the species of a set, or unordered collection. For any set of labels, there will be
exactly one structure, the set itself. \( E \) is irregular, since it has every possible symmetry.
No matter what order the labels are in, we will get the same set.

\subsubsection*{Cycle}
\( C \) is the species of a directed cycle. This is a nonempty loop that goes in a specific
direction. \( C \) is irregular because any permutation that rotates the labels along the loop is
symmetric.

\subsection{Other Operations}
\subsubsection*{Composition}
Intuitively, species composition represents the word ``of''. That is, \( F \circ G \) will be \( F
\)-structures made of \( G \)-structures. To form a structure of \( (F \circ G)[U] \), split \( U \)
into non-empty partitions and send each of those partitions separately to \( G \).  Then, collect
all those \( G \)-structures and send them to \(F \).

For example, \( L \circ E \) is a list of sets, and \( R = 1 + X \bullet (L \circ R) \) is a rose
tree, a tree where the nodes have an unspecified number of children.

\subsubsection*{Differentiation}
Informally, taking the derivative of a species puts a single ``hole'' in the structures, a
distinguished location not holding any data. That is,
\[
  F'[U] = F[U \cup \{*\}]
\]
where \( * \) is a label distinct from all elements of \( U \).~\cite{yorgey} Below are some
examples of derivatives of common species.
\begin{itemize}
  \item
    \( L' = L^2 \).
    If we put a hole in a list, we split it into two lists.
  \item
    \( E' = E \).
    If we put a hole in a set, it does not change the set.
  \item
    \( C' = L \).
    If we put a hole in a cycle, we break the loop into a list.
\end{itemize}

Species differentiation holds its expected properties compared to differentiation in calculus.
\begin{itemize}
  \item \( 1' = 0 \)
  \item \( X' = 1 \)
  \item \( (F + G)' = F + G \)
  \item \( (F \bullet G)' = F \bullet G' + F' \bullet G \)
  \item \( (F \circ G)' = (F' \circ G) \bullet G' \)
\end{itemize}

\subsection{Compound Irregular Species}
\subsubsection*{Powerset}
The species \( E \bullet E \) or \( E^2 \) corresponds to the powerset. Note that the labels given
to this species are split between the two sets, so any given structure will consist of a subset of
the labels and its complement.

\subsubsection*{Partition}
The species \( E \circ E_+ \) is the set of nonempty sets, and thus corresponds to set partitioning.

\subsubsection*{Combination}
Let \( E_m \) be \( E \) restricted to sets of size \( m \). We can represent \( \binom{n}{k} \) by
the species \( E_k \bullet E_{n-k} \)


\section{Generating Functions}
A species \( F \) can be associated with an exponential generating function
\[
  F(x) = \sum_{n \geq 0} f_n \frac{x^n}{n!}
\]
where \( f_n \) is the number of distinct labeled \( F \)-structures of size \( n \).~\cite{yorgey}
Here are the functions corresponding to some species we have seen so far.
{\setlength{\jot}{\baselineskip}
  \begin{align*}
    0(x) & = \frac{0}{0!} x^0 + \frac{0}{0!} x^1 + \cdots = 0               \\
    1(x) & = \frac{1}{0!} x^0                             = 1               \\
    X(x) & = \frac{1}{1!} x^1                             = x               \\
    L(x) & = \sum_{n \geq 0} \frac{n!}{n!} x^n            = \frac{1}{1 - x} \\
    E(x) & = \sum_{n \geq 0} \frac{1}{n!} x^n             = e^x             \\
    C(x) & = \sum_{n \geq 0} \frac{(n - 1)!}{n!} x^n      = -\log{(1 - x)}
  \end{align*}
}

Note that the sum, product, differentiation, and composition of species corresponds to the same
operations on the generating functions, e.g. \( F + G \) is associated with \( F(x) + G(x) \). We
can confirm our earlier intuitions on the derivatives of species with the following remarks.
{\setlength{\jot}{\baselineskip}
  \begin{align*}
    L'(x) & = \frac{1}{{(1 - x)}^2} = {L(x)}^2 \\
    E'(x) & = e^x                   = E(x)     \\
    C'(x) & = \frac{1}{1 - x}       = L(x)
  \end{align*}
}

Although the exponential generating function may be the most obvious and useful series, there are
other power series that can be associated with species such as the type-generating series, cycle
indicator series, and molecular series. These series allow us to demonstrate different properties
and compare species in different ways, such as different kinds of equalities. For instance, two
species may have the same exponential generating series, but a different type-generating series,
helping us to see how they are distinguished.~\cite{bergeron, labelle}

\section{Conclusion}
Combinatorial species are an interesting young branch of combinatorics, with many extensions such as
weighted, multi-sort, virtual, and molecular species.~\cite{yorgey} Species provide computer
scientists a great method for understanding data structures and can be applied to automated testing
to help improve software correctness.

\printbibliography{}

\end{document}
